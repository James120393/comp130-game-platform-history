\documentclass{scrartcl}

\usepackage[hidelinks]{hyperref}
\usepackage[none]{hyphenat}
\usepackage{setspace}
\usepackage{abstract}
\renewcommand{\abstractnamefont}{\normalfont\Large\bfseries}
\doublespace

%Please include a clear, concise, and descriptive title
\title{Atari: The Release And Re-Release of the 2600}

%Please do not change the subtitle
\subtitle{COMP130 - Game Platform History Essay}

%Please put your student ID in the author field
\author{JH182233}

\begin{document}

\maketitle

\begin{abstract}
In business, especially the gaming sector it is important to keep a good balance of profit, popularity and expansion, turning to either one of the options could have detrimental effects on the business. This paper will try to discover why Atari; a once large game developer, seemed to begin looking towards one of these goals more than the other, why Atari re-released the Video Computer System (VCS) with little technological variation several times.
\end{abstract}

\newpage

\section{Introduction}

Atari released the Video Computer System (VCS), later known as the 2600 for its manufacture serial number in 1978\cite{atarii}. Between the years 1978-1991 Atari made substantial profits from the manufacturing of their consoles and games\cite{race}. There is not much publically available information on the different variations of the original VCS, although evidence suggests that there were several of them\cite{age}. Atari began to seem somewhat more inclined to making profit rather than quality products\cite{atarii} this in turn contributed heavily to what is known as the crash of 83\cite{ultimate} was it worth Atari releasing so many versions of what was the same console in terms of technical capabilities?
\newpage

\section{Atari 2600 And Its Many Variations}

Atari may have failed in the later years but the release of the VCS was a very important step for bringing Video Games to the home console. Before the VCS there was Stella, it was the first prototype Atari designed for a console. At a cost of \textdollar100 million to develop, due to this Atari almost went bankrupt so they were not able to manufacture the VCS. That is until Warner purchased Atari for \textdollar28 million which allowed them to begin manufacturing the VCS\cite{101}. Released in 1977 at a cost of \textdollar199 retail no one showed much interest in the VCS so Atari started to panic after all they had poured into their home console it was almost a flop, but with the success of two arcade games Space Invaders and Tanks Atari had the idea of creating a home licensed version of the two. In 1980 the VCS sales soared with the obvious success of the two game releases Atari grossed \textdollar415 million that year\cite{atarii}. It was at this point that it was clear that they had made home gaming a success, everyone had to have a VCS and millions were sold yearly.
\newline
\newline
It was never planned for the 2600 to last more than 2-3 years but due to the success of the console Atari continued to release more versions of the 2600. By 1981 75\% of console sales belonged to the VCS, now called the 2600. Everything was looking good for Atari so they decided to announce the release of a new console The Atari 2600A, due to release in 1982\cite{race}. The functionality of the console was identical to the original 2600, the main difference was all the components that were located on the switchboard were now directly on the motherboard, also known as the Darth Vader the 2600A was also cosmetically different, with a complete black surface and a rearrangement of the switches\cite{manual}. It was around the same time that Atari released the 2800 in Japan, again the functionality was the same as the 2600 but this time taking on a completely different appearance, it did not last long\cite{age}. In light of these two examples it appears that Atari was beginning to try for profits over sustainability.
\newline
\newline
Atari began pushing for more games from the developers, as a result they began making poorly planned licensing agreements. Without correct controls on third-party game development many games that were released on the 2600 were very low quality. Due to the release of more consoles they were beginning to flood the market with products that just were not selling. In 1982 this lead Atari to announced a predicted fall in sales from +50\% to only +10,15\%, this caused a backlash for the company as stock prices fell this chain of events began the fall of Atari as it was known\cite{ultimate,race}. At this time Atari had produced the 5200 designed to be twice as powerful as the 2600, at \textdollar250 retail Atari bought the 2600 down to \textdollar99, as well as this Atari planned on releasing upgrades for the 2600 to improve performance. This was all put to a halt when Warner realised the sudden fall in stock they had to move quickly to stop the hemorrhage of money, so they sold Atari to Jack Tramiel\cite{ultimate,age}. From the evidence it appears that Atari did not plan on the industry to go out of control, party with being too short-sighted as to thinking that no one would start making their own games then continuing without concern to the quality, just the royalties that came from them. 
\newline
\newline
Even through Atari had mostly gone out of business this did not stop them from releasing another variation of the “2600”. In 1986 they released the 2600jr. a console with a different design but the exact capabilities of the original “2600”. Although it was aimed at lower income households selling at \textdollar50 the 2600jr. alongside the NES “Nintendo Entertainment System” the industry seemed to begin recovering, due to this Atari decided to continue producing games for the “2600”, but by 1989 Atari began to realise that their decade long console was reaching the end of its tether. The end came for the “Atari 2600” in 1991 when Atari decided to discontinue their flagship console\cite{atarii,age}. With the crash in 1983 it seemed that Atari was fighting a losing battle, it is quite amazing how they still continued to pull through. But ultimately their mistakes could not be shaken, even today people still make and play games for the 2600.


\section{Conclusion}

When Atari first released the 2600 they were on the track to being the best, for a time they were. But with their unorganised third party development licenses and their seemingly growing thirst for money, they could not keep that position forever. That is all that I can put it down to, Atari wanted more sales so they made new more sleek versions of the same console and hoped for business growth. Unfortunately that is not how it works. Atari had a great, near legendary console that people still adore. If Atari had a greater focus on game quality and console capabilities they may not have fallen, perhaps the crash would have never happened.

\newpage

\bibliographystyle{apalike}
\bibliography{2nd-gen-ref}

\end{document}